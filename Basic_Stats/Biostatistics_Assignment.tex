\documentclass[]{article}
\usepackage{lmodern}
\usepackage{amssymb,amsmath}
\usepackage{ifxetex,ifluatex}
\usepackage{fixltx2e} % provides \textsubscript
\ifnum 0\ifxetex 1\fi\ifluatex 1\fi=0 % if pdftex
  \usepackage[T1]{fontenc}
  \usepackage[utf8]{inputenc}
\else % if luatex or xelatex
  \ifxetex
    \usepackage{mathspec}
  \else
    \usepackage{fontspec}
  \fi
  \defaultfontfeatures{Ligatures=TeX,Scale=MatchLowercase}
\fi
% use upquote if available, for straight quotes in verbatim environments
\IfFileExists{upquote.sty}{\usepackage{upquote}}{}
% use microtype if available
\IfFileExists{microtype.sty}{%
\usepackage{microtype}
\UseMicrotypeSet[protrusion]{basicmath} % disable protrusion for tt fonts
}{}
\usepackage[margin=1in]{geometry}
\usepackage{hyperref}
\hypersetup{unicode=true,
            pdftitle={Biostatistics Assignment},
            pdfauthor={Ayesha Hargey},
            pdfborder={0 0 0},
            breaklinks=true}
\urlstyle{same}  % don't use monospace font for urls
\usepackage{color}
\usepackage{fancyvrb}
\newcommand{\VerbBar}{|}
\newcommand{\VERB}{\Verb[commandchars=\\\{\}]}
\DefineVerbatimEnvironment{Highlighting}{Verbatim}{commandchars=\\\{\}}
% Add ',fontsize=\small' for more characters per line
\usepackage{framed}
\definecolor{shadecolor}{RGB}{248,248,248}
\newenvironment{Shaded}{\begin{snugshade}}{\end{snugshade}}
\newcommand{\KeywordTok}[1]{\textcolor[rgb]{0.13,0.29,0.53}{\textbf{#1}}}
\newcommand{\DataTypeTok}[1]{\textcolor[rgb]{0.13,0.29,0.53}{#1}}
\newcommand{\DecValTok}[1]{\textcolor[rgb]{0.00,0.00,0.81}{#1}}
\newcommand{\BaseNTok}[1]{\textcolor[rgb]{0.00,0.00,0.81}{#1}}
\newcommand{\FloatTok}[1]{\textcolor[rgb]{0.00,0.00,0.81}{#1}}
\newcommand{\ConstantTok}[1]{\textcolor[rgb]{0.00,0.00,0.00}{#1}}
\newcommand{\CharTok}[1]{\textcolor[rgb]{0.31,0.60,0.02}{#1}}
\newcommand{\SpecialCharTok}[1]{\textcolor[rgb]{0.00,0.00,0.00}{#1}}
\newcommand{\StringTok}[1]{\textcolor[rgb]{0.31,0.60,0.02}{#1}}
\newcommand{\VerbatimStringTok}[1]{\textcolor[rgb]{0.31,0.60,0.02}{#1}}
\newcommand{\SpecialStringTok}[1]{\textcolor[rgb]{0.31,0.60,0.02}{#1}}
\newcommand{\ImportTok}[1]{#1}
\newcommand{\CommentTok}[1]{\textcolor[rgb]{0.56,0.35,0.01}{\textit{#1}}}
\newcommand{\DocumentationTok}[1]{\textcolor[rgb]{0.56,0.35,0.01}{\textbf{\textit{#1}}}}
\newcommand{\AnnotationTok}[1]{\textcolor[rgb]{0.56,0.35,0.01}{\textbf{\textit{#1}}}}
\newcommand{\CommentVarTok}[1]{\textcolor[rgb]{0.56,0.35,0.01}{\textbf{\textit{#1}}}}
\newcommand{\OtherTok}[1]{\textcolor[rgb]{0.56,0.35,0.01}{#1}}
\newcommand{\FunctionTok}[1]{\textcolor[rgb]{0.00,0.00,0.00}{#1}}
\newcommand{\VariableTok}[1]{\textcolor[rgb]{0.00,0.00,0.00}{#1}}
\newcommand{\ControlFlowTok}[1]{\textcolor[rgb]{0.13,0.29,0.53}{\textbf{#1}}}
\newcommand{\OperatorTok}[1]{\textcolor[rgb]{0.81,0.36,0.00}{\textbf{#1}}}
\newcommand{\BuiltInTok}[1]{#1}
\newcommand{\ExtensionTok}[1]{#1}
\newcommand{\PreprocessorTok}[1]{\textcolor[rgb]{0.56,0.35,0.01}{\textit{#1}}}
\newcommand{\AttributeTok}[1]{\textcolor[rgb]{0.77,0.63,0.00}{#1}}
\newcommand{\RegionMarkerTok}[1]{#1}
\newcommand{\InformationTok}[1]{\textcolor[rgb]{0.56,0.35,0.01}{\textbf{\textit{#1}}}}
\newcommand{\WarningTok}[1]{\textcolor[rgb]{0.56,0.35,0.01}{\textbf{\textit{#1}}}}
\newcommand{\AlertTok}[1]{\textcolor[rgb]{0.94,0.16,0.16}{#1}}
\newcommand{\ErrorTok}[1]{\textcolor[rgb]{0.64,0.00,0.00}{\textbf{#1}}}
\newcommand{\NormalTok}[1]{#1}
\usepackage{graphicx,grffile}
\makeatletter
\def\maxwidth{\ifdim\Gin@nat@width>\linewidth\linewidth\else\Gin@nat@width\fi}
\def\maxheight{\ifdim\Gin@nat@height>\textheight\textheight\else\Gin@nat@height\fi}
\makeatother
% Scale images if necessary, so that they will not overflow the page
% margins by default, and it is still possible to overwrite the defaults
% using explicit options in \includegraphics[width, height, ...]{}
\setkeys{Gin}{width=\maxwidth,height=\maxheight,keepaspectratio}
\IfFileExists{parskip.sty}{%
\usepackage{parskip}
}{% else
\setlength{\parindent}{0pt}
\setlength{\parskip}{6pt plus 2pt minus 1pt}
}
\setlength{\emergencystretch}{3em}  % prevent overfull lines
\providecommand{\tightlist}{%
  \setlength{\itemsep}{0pt}\setlength{\parskip}{0pt}}
\setcounter{secnumdepth}{0}
% Redefines (sub)paragraphs to behave more like sections
\ifx\paragraph\undefined\else
\let\oldparagraph\paragraph
\renewcommand{\paragraph}[1]{\oldparagraph{#1}\mbox{}}
\fi
\ifx\subparagraph\undefined\else
\let\oldsubparagraph\subparagraph
\renewcommand{\subparagraph}[1]{\oldsubparagraph{#1}\mbox{}}
\fi

%%% Use protect on footnotes to avoid problems with footnotes in titles
\let\rmarkdownfootnote\footnote%
\def\footnote{\protect\rmarkdownfootnote}

%%% Change title format to be more compact
\usepackage{titling}

% Create subtitle command for use in maketitle
\newcommand{\subtitle}[1]{
  \posttitle{
    \begin{center}\large#1\end{center}
    }
}

\setlength{\droptitle}{-2em}

  \title{Biostatistics Assignment}
    \pretitle{\vspace{\droptitle}\centering\huge}
  \posttitle{\par}
    \author{Ayesha Hargey}
    \preauthor{\centering\large\emph}
  \postauthor{\par}
      \predate{\centering\large\emph}
  \postdate{\par}
    \date{May 9, 2019}

\usepackage{setspace}\doublespacing

\begin{document}
\maketitle

\subsection{Introduction}\label{introduction}

The common cuckoo (\emph{Cuculus canorus}) is a charismatic passerine
bird and the keystone example of brood parasites in animals. The mother
bird lays mimicking eggs into nests of small songbirds, and thereafter,
the cuckoo hatchling eliminates the breeding success of the host by
evicting all other eggs and offspring from the nest (Moskát and Hauber,
2007). This host is then responsible for the parental care of these
genetically unrelated young. Many of these host species have evolved
defense mechanisms to prevent or reduce the likelihood of raising a
parasitic egg such as through ejection or the desertion of brood (Moskát
and Hauber, 2007). This then selects for improved egg mimicry by the
cuckoo (Marchetti, 2000).

However, this identification is highly variable as it hinges on the
evolutionary history of the host species. Egg identification and
discrimination, thus, most likely has a genetic basis (Martin-Galvez et
al., 2006). The current prevailing theory of recognition is that hosts
compare their own eggs with the parasitics and reject what looks
different (Marchetti, 2000). Another theory relies on the concept of
learning, a facet of which includes host birds memorize the pattern in
which they lay their eggs (Hauber, Sherman and Paprika, 2000).
Ultimately, these systems are strongly influenced by the extent of the
mimetic similarity of the parasite egg to the host egg, with more
accurate eggs being rejected at lower rates (Røskaft et al., 1991).
There are costs to the host for egg rejection, most notably that
mistaken identification can result in their own egg or brood being
harmed or deserted, which could cause costs which outweigh the benefits
(Davies, Brooke and Kacelnik, 1996).

Cuckoo birds present an opportunity to be assessed as an important
indicator of avian biodiversity. It is a bird that is monitored with
minimal difficult, has a global distribution and its distinctive call
ensures easy identification (Haest, 2019). This further highlights the
importance of studying the behavioural and reproductive habits of these
birds.

This study is limited due to its comparatively small sample size but
provides a basis for future research into the field on the egg
morphology of cuckoo birds. This study was conducted using 6 of the most
common host birds which represents the largest proportion of species.

The purpose of this investigation is to determine if there is a
relationship between the sizes of cuckoo eggs and the species of
foster-parent. The length and breadth of cuckoo eggs were measured, with
an additional comparison between the eggs of the host species. It is
predicted that such a difference does exist.

\subsection{Methods}\label{methods}

\subsection{Results}\label{results}

There is a clear visual distinction between cuckoo egg dimensions and
the species of host parent. Figure 1 and 2 clearly demonstrate this,
with wren being noticably smaller than the other species.

\includegraphics{Biostatistics_Assignment_files/figure-latex/cuckoo egg length boxplot-1.pdf}

Figure 1: Diagram of cuckoo egg length corresponding to species of
host-parent

\includegraphics{Biostatistics_Assignment_files/figure-latex/cuckoo egg breadth boxplot-1.pdf}

Figure 2: Diagram of cuckoo egg breadth corresponding to species of
host-parent

\includegraphics{Biostatistics_Assignment_files/figure-latex/anova-1.pdf}

Figure 3: Tukey Analysis of Egg Length and Egg Breadth

The largest difference occurs betwwen the Meadow Pipit and Wren, the
Pied Wagtail and Wren, and the Tree Pipit and Wren. There is minimal
difference among the bigger birds. In particular, there is a far greater
significance in length than of breadth. Egg morphology in cuckoo birds
is highly variable.

\begin{Shaded}
\begin{Highlighting}[]
\NormalTok{pearson_cuckoos <-}\StringTok{ }\KeywordTok{cor.test}\NormalTok{(}\DataTypeTok{x =}\NormalTok{ cuckoos}\OperatorTok{$}\NormalTok{length, cuckoos}\OperatorTok{$}\NormalTok{breadth)}
\NormalTok{r_print <-}\StringTok{ }\KeywordTok{paste0}\NormalTok{(}\StringTok{"r = 0.5"}\NormalTok{)}
\NormalTok{correlation_cuckoos <-}\StringTok{ }\KeywordTok{ggplot}\NormalTok{(}\DataTypeTok{data =}\NormalTok{ cuckoos, }\KeywordTok{aes}\NormalTok{(}\DataTypeTok{x =}\NormalTok{ length, }\DataTypeTok{y =}\NormalTok{ breadth)) }\OperatorTok{+}
\StringTok{  }\KeywordTok{geom_smooth}\NormalTok{(}\DataTypeTok{method =} \StringTok{"lm"}\NormalTok{, }\DataTypeTok{colour =} \StringTok{"slategray2"}\NormalTok{, }\DataTypeTok{se =}\NormalTok{ F) }\OperatorTok{+}
\StringTok{  }\KeywordTok{geom_point}\NormalTok{(}\DataTypeTok{colour =} \StringTok{"tomato2"}\NormalTok{) }\OperatorTok{+}
\StringTok{  }\KeywordTok{geom_label}\NormalTok{(}\DataTypeTok{x =} \DecValTok{20}\NormalTok{, }\DataTypeTok{y =} \FloatTok{17.3}\NormalTok{, }\DataTypeTok{label =}\NormalTok{ r_print) }\OperatorTok{+}
\StringTok{  }\KeywordTok{labs}\NormalTok{(}\DataTypeTok{x =} \StringTok{"Egg length (mm)"}\NormalTok{, }\DataTypeTok{y =} \StringTok{"Egg breadth (mm)"}\NormalTok{) }\OperatorTok{+}
\StringTok{  }\KeywordTok{theme_pubclean}\NormalTok{() }
\NormalTok{correlation_cuckoos}
\end{Highlighting}
\end{Shaded}

\includegraphics{Biostatistics_Assignment_files/figure-latex/correlation-1.pdf}

Figure 4: Pearson's correlation of egg length to egg breadth (r = 0.5)

In an attempt to determine whether there was a relationship between egg
breadth and egg length, a Pearson's correlation test was done, which
yieled the result as displayed in Figure 4. With a value of r = 0.5,
there is a slightly strong correlation between egg length and egg
breadth, suggesting that it proportionally increased. \#\# Discussion

\subsection{Including Plots}\label{including-plots}

You can also embed plots, for example:

\includegraphics{Biostatistics_Assignment_files/figure-latex/pressure-1.pdf}

Note that the \texttt{echo\ =\ FALSE} parameter was added to the code
chunk to prevent printing of the R code that generated the plot.


\end{document}
